This chapter will give a brief overview of the theoretical background about PV
systems (Land Based and Floating) behind the methods and algorithms used in this
thesis. It will also give an overview of the data used, and how it is structured.
The theory will be explained in a way that is easy to understand, and will not
go into too much detail.

\section{Land-based photovoltaic (PV) systems}
Land-based photovoltaic (PV) systems are solar energy systems installed on the
ground or rooftops, utilizing solar panels to convert sunlight into electricity.
These systems harness solar energy for residential, commercial, and utility-scale
applications. Key components include solar panels (typically silicon-based),
inverters for DC--AC conversion, mounting structures, and electrical connections.
Mounting structures ensure panels are securely positioned for optimal sunlight
capture. These systems can vary from small residential setups to large-scale solar
farms feeding the grid. They play a crucial role in promoting renewable energy
adoption, reducing greenhouse gas emissions, and supporting a sustainable energy
future.

According to the IEA, land-based PV is one of the fastest-growing sources of renewable
energy worldwide, benefiting from low operating costs, scalability, and
versatile siting---including rooftops, open fields, and brownfield sites. Advances
in solar technology have improved efficiency and lowered costs, enhancing
economic viability. Recent studies support these advantages: Belloni et al.\ (2023)
demonstrated how machine-learning optimization can improve O\&M efficiency;
Bindi et al.\ (2022) documented dramatic cost reductions, making PV is
competitive with fossil fuels, and environmental assessments indicate minimal land-use
impact.

\section{Floating PVs}
Floating photovoltaic (PV) systems, or floating solar, are innovative installations
deployed on bodies of water such as lakes, reservoirs, and ponds. These systems
consist of solar panels mounted on floating platforms or pontoons, allowing them
to harness sunlight while floating on the water's surface. ()

Floating PV systems offer several advantages over traditional land-based
installations:
\begin{itemize}
  \item \textbf{Reduced land use}: They do not occupy terrestrial land and can be
    deployed over existing water bodies---including reservoirs, irrigation ponds,
    and brownfields.

  \item \textbf{Natural cooling and higher efficiency}: Water cooling reduces operating
    temperatures, increasing electrical output by 5--15\%, with some field tests
    reporting up to 11\% improvements.

  \item \textbf{Water evaporation suppression}: Shading from panels reduces reservoir
    evaporation by 42--83\% in empirical tests---e.g., a 130 kW installation in Brazil
    cut evaporation by 60\%.

  \item \textbf{Higher energy yield and LCA performance}: FPV systems can outperform
    ground-mounted configurations and feature shorter energy payback times (approximately
    1.3 years) and lower lifecycle GHG emissions.

  \item \textbf{Synergy with hydropower}: They can integrate with dams and transmission
    infrastructure, enhancing grid flexibility.
\end{itemize}

\section{Anomalies types in Floating PVs}

Like their land-based counterparts, floating photovoltaic (FPV) systems can
experience various types of anomalies that affect their performance and
efficiency. These anomalies include soiling, algae growth, shading, and
structural issues. Detecting these anomalies early is crucial for maintaining an
optimal system performance and ensuring the longevity of the floating PV system.

\section{Anomalies Types in Land-based PVs}
PV systems can experience various anomalies affecting their performance and efficiency.
These anomalies include hot spots, cell cracks, soiling, shading, and
degradation. Detecting these anomalies early is crucial for maintaining optimal system
performance.

\noindent
\textbf{Hot Spots:} Hot spots occur when a portion of a solar panel becomes
significantly hotter than the surrounding areas. This can happen due to partial shading,
cell mismatch, or manufacturing defects. Hot spots can lead to reduced
efficiency and even permanent damage to the solar cells if not addressed
promptly.

\noindent
\textbf{Cell Cracks:} Cell cracks are physical damage to the solar cells that can
occur during manufacturing, transportation, or installation. These cracks can
reduce the electrical output of the affected cells and may lead to further
degradation over time. Detecting cell cracks early is essential to prevent further
damage and ensure the longevity of the solar panel.
\noindent
\textbf{Soiling:} Soiling refers to the accumulation of dirt, dust, and other debris
on the surface of solar panels. This layer of dirt can block sunlight from
reaching the solar cells, significantly reducing their efficiency. Regular
cleaning and maintenance are necessary to prevent soiling from impacting the
performance of the PV system.
\noindent
\textbf{Shading:} Shading occurs when objects such as trees, buildings, or other
structures block sunlight from reaching the solar panels. Even partial shading can
cause significant drops in energy production, as solar panels are designed to
operate optimally in direct sunlight. Identifying and mitigating shading issues is
crucial for maintaining the efficiency of PV systems.
\noindent
\textbf{Degradation:} Over time, solar panels can experience degradation due to
exposure to environmental factors such as UV radiation, temperature fluctuations,
and moisture. This degradation can lead to a gradual decline in the performance of
the solar panels. Regular monitoring and maintenance are necessary to assess the
condition of the panels and address any degradation issues promptly.

\section{Power loss in PV systems}
Power loss in photovoltaic (PV) systems refers to the reduction in electrical
output due to various factors, including shading, soiling, temperature effects,
and anomalies such as hot spots and cell cracks. These losses can significantly impact
the overall efficiency and performance of the PV system, leading to lower energy
production and increased operational costs. Understanding the causes of power loss
is crucial for optimizing the design and operation of PV systems, ensuring that
they operate at their maximum potential. By implementing effective monitoring and
maintenance strategies, operators can minimize power loss and enhance the
overall efficiency of photovoltaic systems.

\section{Anomalies Detection Methods}
Anomaly detection methods in photovoltaic (PV) systems are essential for
identifying and addressing issues that can impact the performance and efficiency
of solar panels. These methods leverage various techniques, including
statistical analysis, machine learning, and computer vision, to automatically detect
anomalies such as hot spots, cell cracks, soiling, shading, and degradation. By
employing these methods, operators can ensure timely maintenance and repairs,
ultimately enhancing the reliability and longevity of PV systems.

\subsection{Visual Inspection}

Visual inspection is a traditional method for detecting anomalies in photovoltaic(PV)
systems. It involves manually examining solar panels for visible defects such as
cracks, discoloration, and soiling. While visual inspection can be effective for
identifying obvious issues, it is often time-consuming and labor-intensive,
especially for large-scale PV installations. Additionally, human inspectors may
overlook subtle anomalies that could affect the performance of the solar panels.
To address these limitations, automated visual inspection techniques have been
developed, leveraging computer vision and machine learning algorithms to analyze
images of solar panels and detect anomalies with higher accuracy and efficiency.
These automated systems can process large volumes of data quickly, reducing the need
for manual labor and increasing the speed of anomaly detection in PV systems.

\setlength{\parskip}{0.5em plus 0.2em minus 0.2em}
\setlength{\parindent}{0pt}

\subsection{IR Thermography}

IR thermography, or infrared thermography, is a non-destructive testing technique
that uses infrared cameras to detect and visualize temperature variations on the
surface of objects. In the context of photovoltaic (PV) systems, IR thermography
is employed to identify anomalies such as hot spots, cell cracks, and other thermal
issues that can affect the performance of solar panels. By capturing thermal
images, operators can quickly assess the condition of PV systems, enabling them
to detect potential problems early and take corrective actions to maintain
optimal performance.

\subsection{EL Imaging}

Electroluminescence (EL) imaging is a technique used to visualize the internal structure
of photovoltaic (PV) cells by capturing the light emitted when an electric current
is passed through the cells. This method is particularly useful for detecting
anomalies such as micro-cracks, shunts, and other defects that may not be
visible through traditional visual inspection methods. EL imaging provides a detailed
view of the PV cells, allowing operators to identify and address issues that can
affect the performance and efficiency of solar panels. By employing EL imaging,
operators can ensure the reliability and longevity of PV systems, ultimately enhancing
their energy production capabilities.

\section{Computer Vision}
Computer vision is a field of artificial intelligence that focuses on enabling machines
to interpret and understand visual information from the world. It involves the
development of algorithms and models that can process, analyze, and extract
meaningful information from images and videos. In the context of solar PV systems,
computer vision techniques can be employed to automatically detect and classify
anomalies in solar panels, such as cracks, soiling, and shading. By leveraging
computer vision, it is possible to automate the inspection process, reducing the
need for manual labor and increasing the efficiency of anomaly detection in
solar PV systems.