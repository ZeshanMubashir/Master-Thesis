% The subsections written are only suggestions, to display how sections and subsections may look for your thesis

This chapter provides an overview of the research topic, including the
background, motivation, scope, objectives, and significance of the study. It also
outlines the research questions and the thesis structure.

\section{Background and Motivation}
\sloppy

The motivation for this research is to address the growing concern about
efficiency and reliability in solar photovoltaic (PV) systems. As renewable energy
adoption accelerates globally, ensuring optimal performance of solar
installations becomes increasingly critical for energy security and climate

% Adding \raggedright to improve layout
\raggedright. Additionally, there is a growing interest in the use of machine learning
(ML) and deep learning (DL) techniques for the detection of anomalies in solar PVs.
These techniques have shown promise in various domains, including image processing,
natural language processing, and anomaly detection. These computational
approaches can analyze complex patterns in data that might be imperceptible to
human inspectors, potentially improving the accuracy and speed of fault detection.
ML and DL models can be trained on large datasets of solar PV images to
recognize visual signatures of different anomalies, from microcracks and hotspots
to soiling and delamination.

The International Energy Agency (IEA) latest report shows that renewable energy capacities
realized in 2022 reached a record 295 GW, with solar PV accounting for nearly 60\%
of this growth. The global installed capacity of solar PV systems has surpassed
1,000 GW, making it one of the fastest-growing energy sources worldwide. However,
the performance of solar PV systems can be significantly affected by various anomalies,
such as physical defects, soiling, and electrical faults. These anomalies can lead
to reduced energy yield, increased maintenance costs, and even system failures.
Traditional methods of anomaly detection in solar PVs often rely on manual
inspection, which is time-consuming, labor-intensive, and prone to human error.
As a result, there is a pressing need for automated anomaly detection systems that
can efficiently and accurately identify issues in solar PV installations.

Traditional inspection methods rely on manual processes that are time-consuming,
labor-intensive, and often inconsistent. Automated anomaly detection using
machine learning and computer vision presents a promising alternative, offering potential
for continuous monitoring, early fault detection, and reduced operational costs.
However, significant challenges remain in developing robust models that can operate
reliably across diverse environmental conditions, panel types, and anomaly categories.

This research aims to bridge these gaps by systematically investigating the complete
pipeline from data preprocessing to model deployment, with particular emphasis
on practical implementation challenges in real-world solar PV installations.

\section{Scope}

The scope of this research is to investigate the use of machine learning (ML) and
deep learning (DL) techniques for the detection of anomalies in solar PVs. The
research will focus on the development of algorithms that can be used to
automatically detect anomalies in solar PVs, and the evaluation of these algorithms
using real-world data. The research will also investigate the use of data
preprocessing techniques, such as data augmentation to improve the performance of
the deep learning algorithms.

\section{Objectives}

Land-based and floating solar PV systems are increasingly deployed to meet
global energy demands sustainably. However, the efficiency and reliability of
these systems can be compromised by various anomalies, such as physical defects,
soiling, and electrical faults. This research aims to develop a comprehensive
framework for automated anomaly detection in solar PV systems using machine
learning and computer vision techniques. The primary objectives of this research
are:

\begin{itemize}
    \item To explore and implement various data preprocessing techniques,
        including image augmentation and data cleaning, to enhance the performance
        of machine learning and deep learning models for solar PV anomaly detection.

    \item To evaluate and compare the performance of different machine learning and
        deep learning architectures in detecting and classifying specific types of
        anomalies in solar PV systems.

    \item To develop methodological frameworks that address challenges such as class
        imbalance, data scarcity, and environmental variability in computer vision-based
        solar PV anomaly detection systems.
\end{itemize}

\section{Research Questions}

This thesis aims to advance the field of automated anomaly detection in solar photovoltaic
systems through the application of machine learning and computer vision techniques.
Specifically, the research addresses the following questions:

\begin{itemize}
    \item \textbf{Research Question 1:} How do various data preprocessing
        techniques, particularly image augmentation, influence the performance
        of deep learning models for solar PV anomaly detection?

    \item \textbf{Research Question 2:} Which deep learning (ViT and CNN variants)
        architectures demonstrate superior performance metrics (accuracy,
        precision, recall, F1-score) for detecting and classifying specific types
        of anomalies in solar PV systems?

    \item \textbf{Research Question 3:} What methodological frameworks and
        technical approaches best address the challenges of class imbalance, data
        scarcity in computer vision-based solar PV anomaly detection systems?
\end{itemize}

\section{Significance of the Study}
The significance of this study lies in its potential to enhance the efficiency
and reliability of solar photovoltaic systems through advanced anomaly detection
techniques. By leveraging machine learning and computer vision, this research
aims to provide a robust framework for early fault detection, which can lead to reduced
maintenance costs, increased energy yield, and extended system lifespan. This
study contributes to the broader field of renewable energy by addressing the
critical need for automated, reliable, and efficient monitoring solutions in
solar PV installations. The findings are expected to have practical implications
for solar energy operators, policymakers, and researchers, facilitating the
wider adoption of solar technologies and supporting global efforts towards
sustainable energy transition.

Solar PVs are a renewable energy source that can be used to generate electricity.
They are widely used in residential and commercial applications, and their use
is expected to grow in the coming years.

\section{Thesis Structure}

The structure of this thesis is organized as follows:
\begin{itemize}
    \item \textbf{Chapter 1: Introduction} - Provides an overview of the
        research topic, including background, motivation, scope, objectives, and
        significance.

    \item \textbf{Chapter 2: Literature Review and Theory} - Reviews existing
        literature on solar PV anomaly detection, deep learning techniques, and
        computer vision applications in renewable energy.

    \item \textbf{Chapter 3: Methodology} - Describes the research design, including
        data collection, preprocessing techniques, model development, and
        evaluation metrics.

    \item \textbf{Chapter 4: Results and Discussion} - Presents the results of
        the experiments conducted and discusses their implications in the
        context of the research questions.

    \item \textbf{Chapter 5: Conclusion and Future work} - Summarizes the key
        findings, contributions of the research, and suggests directions for
        future work.
\end{itemize}