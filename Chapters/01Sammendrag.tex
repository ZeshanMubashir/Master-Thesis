\noindent

Solenergi er en fremvoksende fornybar energikilde som har fått betydelig
oppmerksomhet de siste årene. Den økende etterspørselen etter bærekraftige
energiløsninger har ført til utforskning av ulike solenergi-teknologier, inkludert
landbaserte og flytende fotovoltaiske (PV) systemer. Med utbredt bruk er det
imidlertid behov for å evaluere ytelsen til disse systemene under forskjellige
miljøforhold for å bestemme deres effektivitet, kostnadseffektivitet og
miljøpåvirkning.

Hovedmålet med denne studien er å utvikle dyplæringsmodeller for å oppdage
og klassifisere defekter i fotovoltaiske (PV) systemer, med spesielt fokus på
landbaserte og flytende PV-systemer. Forskningen tar sikte på å addressere
utfordringene knyttet til ytelsen til disse systemene, særlig når det gjelder
deres effektivitet, kostnadseffektivitet og miljøpåvirkning. Studien vil benytte
simuleringsverktøy for å modellere ytelsen til landbaserte og flytende PV-systemer
under ulike miljøforhold, som solinnstråling, temperaturvariasjoner og påvirkningen
av vannmasser på effektiviteten til flytende PV-systemer.
