\noindent

Solar energy is an emerging renewable energy source that has gained significant
attention in recent years due to increasing demand for sustainable energy solutions.
This has led to the exploration of various photovoltaic (PV) technologies, including
land-based and floating photovoltaic systems. However, with widespread adoption
comes the critical need for effective monitoring and maintenance strategies to
ensure optimal performance, efficiency, and longevity of these installations.

This study focuses on developing advanced deep learning models for automated
detection and classification of anomalies in photovoltaic systems using thermal
imaging data. The research addresses the growing need for intelligent inspection
systems that can identify various types of defects including hot spots, cell
cracks, diode failures, soiling, vegetation blocking, shadowing, and offline
modules. Using the InfraredSolarModules dataset containing over 20,000 thermal
images representing 11 different anomaly types, multiple deep learning architectures
were evaluated, including traditional convolutional neural networks (ResNet50,
VGG16, EfficientNet-B0, Xception) and modern Vision Transformers (ViT-B32, DeiT-B16).

The methodology encompasses comprehensive evaluation across three classification
scenarios: binary classification (anomaly vs. no anomaly), 11-class classification
(anomaly types only), and 12-class classification (including normal conditions).
Advanced visualization techniques, including t-SNE dimensionality reduction and
Grad-CAM activation mapping, were employed to provide insights into learned
feature representations and model decision-making processes.

Results demonstrate that Vision Transformer architectures achieved superior
performance compared to traditional CNNs, with binary classification achieving
over 98\% accuracy in distinguishing between normal and anomalous conditions.
The study reveals varying performance across different anomaly types, with some
classes achieving F1-scores above 95\% while others present more challenging
detection scenarios due to class imbalance and subtle thermal signatures.

This research contributes to the advancement of automated solar panel inspection
systems and provides a foundation for implementing intelligent maintenance
strategies in photovoltaic installations. The findings support the development
of cost-effective, scalable monitoring solutions that can enhance the reliability
and efficiency of solar energy systems, ultimately contributing to the broader
adoption of sustainable energy technologies.