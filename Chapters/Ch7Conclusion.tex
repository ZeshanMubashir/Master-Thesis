\label{chap:conclusion}

Give a concise summary of your research and finding here, and include a short
summary of any future work as well.

\begin{itemize}
    \item Summary of research

    \item Key findings

    \item Future work
\end{itemize}

\textbf{Research Question 1} 

\textbf{Research Question 2}

\textbf{Research Question 3}

Background information on the research topic: The research topic is centered
around image processing, specifically focusing on noise reduction techniques and
their impact on image quality. The study explores various algorithms and their
effectiveness in enhancing images while preserving important details.

\section{Future Work and Recommendations}

Based on the results and limitations of this study, several avenues for future
research are recommended:
\begin{itemize}
    \item \textbf{Exploring More Architectures:} Future work could expand the
        analysis to include other advanced architectures, such as hybrid CNN-Transformer
        models or more recent state-of-the-art object detection models like YOLO
        or DETR, to not only classify but also localize the anomalies.

    \item \textbf{Real-World Deployment and Testing:} The next logical step is
        to test the trained model on a live video feed from a drone-mounted
        thermal camera to evaluate its real-time performance and robustness in a
        dynamic field environment.

    \item \textbf{Addressing Class Imbalance:} Further investigation into
        advanced techniques for handling class imbalance, such as cost-sensitive
        learning or generative adversarial networks (GANs) for synthetic data generation,
        could improve performance on rare anomaly classes.\cite{he2016deep}

    \item \textbf{Transfer Learning from Other Domains:} Exploring transfer
        learning from related domains, such as medical imaging or satellite imagery
        analysis, could provide additional insights for improving anomaly
        detection performance in solar panel inspection tasks.

    \item \textbf{Multi-Modal Data Fusion:} Future research could explore fusing
        thermal data with corresponding visual-spectrum (RGB) images to provide
        richer information to the model, potentially improving classification accuracy
        for ambiguous cases like shadowing versus soiling.

    \item \textbf{Longitudinal Studies:} Conducting longitudinal studies to
        monitor the performance of solar panels over time, using the developed models
        to detect changes in performance and predict future anomalies, could
        provide valuable insights into the long-term reliability of solar energy
        systems.

    \item \textbf{Integration with IoT Systems:} Future work could focus on
        integrating the anomaly detection system with IoT platforms for real-time
        monitoring and alerting, enabling proactive maintenance strategies in solar
        energy systems.

    \item \textbf{User Interface Development:} Developing a user-friendly
        interface for maintenance teams to visualize detected anomalies and
        access detailed reports could enhance the practical utility of the
        developed system, making it easier to implement in real-world scenarios.

        Future work could also explore multi-modal approaches that combine thermal \cite{deline2010partially}
        imaging with other data sources, such as visual spectrum images or
        electrolumsinescence images to improve the model performance and
        diversity to captures a wider range of anomalies.
\end{itemize}