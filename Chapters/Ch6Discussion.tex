% Start with an introductory paragraph for the chapter.
In this chapter, we discuss the results presented in the previous chapter,
interpreting their significance and implications. We also acknowledge the limitations
of the study, provide a concluding summary, and suggest directions for future research.

\section{Discussion}


What will be the impact of the results?

The results of our experiments demonstrate the effectiveness of [Your Model, e.g.,
the fine-tuned ViT model] in detecting anomalies in infrared images of solar panels.
The high F1-score of [Your Score] for the [Best Class] class suggests that
\cite{ibrahim2022machine} the model is particularly adept at identifying [Type of Anomaly].

This performance can be attributed to [Reason, e.g., the Vision Transformer's
ability to capture global context within the image...]. When compared to the baseline
[e.g., ResNet50] model, our proposed architecture achieved a [e.g., 5%] improvement in overall accuracy. This finding aligns with recent studies by [Author, Year], which also highlight the advantages of transformer-based models in specialized computer vision tasks.

\subsection{Implications of the Results}
The findings of this research have several practical implications for the solar energy
industry. The successful application of [Your Technology] for automated anomaly
detection can lead to:
\begin{itemize}
    \item \textbf{Improved Maintenance Efficiency:} Automated systems can
        rapidly scan thousands of panels, allowing maintenance teams to focus their
        efforts on confirmed defects, thereby reducing operational costs.

    \item \textbf{Increased Energy Yield:} Early detection of issues like \cite{}
        soiling, cracking, or faulty diodes prevent long-term power loss and
        maximizes the energy output of solar farms.

    \item \textbf{Enhanced Safety:} Identifying potential hazards such as hot
        spots can prevent catastrophic failures and improve the overall safety
        of solar installations.
\end{itemize}
From a research perspective, this work contributes to the growing body of
literature on applying deep learning to renewable energy infrastructure, demonstrating
that models pre-trained on natural images can be successfully fine-tuned for highly
specialized thermal imaging tasks.

\section{Limitations}

While this study provides valuable insights, it is important to acknowledge its limitations.
\begin{itemize}
    \item \textbf{Dataset Specificity:} The models were trained and evaluated
        exclusively on the \textit{IR} dataset. Their performance may vary on
        images captured with different thermal cameras, under different environmental
        conditions, or from different types of solar panels not represented in
        the dataset.

    \item \textbf{Class Imbalance:} Despite using data augmentation, the significant
        imbalance in the dataset, especially the underrepresentation of classes
        like "Soiling" and "Hot-Spot-Multi," may have limited the model's ability
        to learn robust features for these rare anomalies.

    \item \textbf{Computational Resources:} The scope of hyperparameter tuning
        was constrained by the available computational resources. A more extensive
        search could potentially yield a model with even higher performance.
\end{itemize}

\section{Conclusion}

This research set out to develop and evaluate deep learning models for the
automated detection of anomalies in solar panels using infrared imagery. We successfully
implemented and compared several architectures, including CNNs and Vision
Transformers, on the publicly available \textit{IR} dataset.